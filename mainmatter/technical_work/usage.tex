\section{Usage}
\label{sec:Usage}
The utilisation of the implemented programs is quite straightforward. No additional command needs to be entered, because every program has a build file. For example, with the implementation in the Go language, the initial arborescence (the one before compilation) is the following :

\begin{figure}[H]
\dirtree {%
.1 ..
.1 src.
	.2 levenshtein.
		.3 dynamic\_levenshtein\_distance\_technique.go.
		.3 levenshtein\_distance\_calculator.go.
		.3 levenshtein\_distance\_technique.go.
		.3 recursive\_levenshtein\_distance\_technique.go.
	.2 levenshtein\_test.
    	.3 levenshtein\_test.go.
    .2 main.
		.3 main.go.
}
\end{figure}

After setting the \verb+$GOPATH+ environment variable (needed by the Go compiler to find the module) to the project root folder (\verb+gc-levenshtein/go/+), you have to compile the program by executing the command \verb+go install main+.
\par Two new folders will be generated : one for the packages (\verb+pkg/+) and another one for the binaries (\verb+main+). In our context, the package is \verb+levenshtein+ and the binary is \verb+main+.

\begin{figure}[H]
\dirtree {%
.1 ..
.1 bin.
	.2 main.
.1 pkg.
	.2 linux\_amd64.
		.3 levenshtein.a.
.1 src.
	.2 levenshtein.
		.3 dynamic\_levenshtein\_distance\_technique.go.
		.3 levenshtein\_distance\_calculator.go.less
		.3 levenshtein\_distance\_technique.go.
		.3 recursive\_levenshtein\_distance\_technique.go.
	.2 levenshtein\_test.
    	.3 levenshtein\_test.go.
    .2 main.
		.3 main.go.
}
\end{figure}

To run the program, just launch the binary file with the name of the file containing the words as an argument. For example : \verb+./bin/main ../dictionary_EN.txt+. The following output (truncated) will be produced in the standard output :

\begin{lstlisting}[breaklines]

...

Word1 : micropipette
Word2 : microprocessing
Recursive distance : 8
Dynamic distance : 8

Word1 : microprocessor
Word2 : microprocessors
Recursive distance : 1
Dynamic distance : 1

Word1 : microprogram
Word2 : microprogrammed
Recursive distance : 3
Dynamic distance : 3

Word1 : microprogramming
Word2 : microradiographical
Recursive distance : 9
Dynamic distance : 9

Word1 : microradiographically
Word2 : microradiography
Recursive distance : 5
Dynamic distance : 5

Word1 : micros
Word2 : microscope
Recursive distance : 4
Dynamic distance : 4

...

\end{lstlisting}